\section{Etat à la fin de stage}

\paragraph{Fonctionalité}
La première version de l'application opérationnelle est sur un serveur de production.
Jusqu'à présent, l'application a été déployée plus de 70 fois.
Après mon stage, cette application va devenir un projet communautaire,
les développeurs de VSCT peuvent proposer des contributions,
les tests automatiques et déploiement automatique seront appliqués pour assurer le bon déploiement.

\paragraph{Démo}
Une démo au sein de l'équipe des intégrateurs est déja réalisé.
Ce n'était pas encore une version complète, mais les intégrateurs ont compris le but de ce projet et ont proposé leurs idées d'amélioration.

Une démo global était prévue au début juillet pour recueillir de nouveau besoins.
Elle a été retardé et replanifié au début août pour synchroniser avec un autre projet de stage.

\paragraph{Bilan}
Par rapport à l'objectif, le travail réalisé a couvert l'ensemble des objectifs, mais il reste des éléments suivants :
\begin{itemize}
  \item Des tests fonctionnels pour compléter l'automatisation de test ;
  \item Un perimétrage plus précis des caches ;
  \item Le traitement de cas exceptionnel lié à certaines applications dans Nexus ;
  \item D'autres APIs.
\end{itemize}

\clearpage

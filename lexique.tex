\chapter{Lexique}

\section{Lexique Général}

\paragraph{EPIC: }
\label{lexi:epic}
Établissement public à caractère industriel et commercial.

\paragraph{Nexus: }
\label{lexi:nexus}
Nexus est un entrepôt de binaires.

\paragraph{Lucène: }
\label{lexi:lucene}
Nexus est une interface api fournie par Nexus.

\paragraph{Kanban: }
\label{lexi:kanban}
Kanban est une méthode de gestion des connaissances relatives au travail,
qui met l’accent sur une organisation de type Juste-à-temps en fournissant l'information ponctuellement aux membres de l'équipe afin de ne pas les surcharger.
Dans cette approche, le processus complet de l'analyse des tâches jusqu’à leur livraison au client est consultable par tous les participants,
chacun prenant ses tâches depuis une file d'attente.

\paragraph{Devops: }
\label{lexi:devops}
Le devops est un mouvement visant à l'alignement de l'ensemble des équipes du système d'information sur un objectif commun,
à commencer par les équipes de dev ou dev engineers chargés de faire évoluer le système d'information et les ops ou ops engineers responsables des infrastructures (exploitants, administrateurs système, réseau, bases de données,...).
Ce qui peut être résumé par : travailler ensemble pour produire de la valeur pour l'entreprise.

\paragraph{Post-it theory: }
\label{lexi:post_it_theory}
C’est une méthode « hyper procédurière » de conduite de projet agile.
« Hyper procédurière » car elle s’appuie sur un ensemble de rituels tels que le daily meeting, la démonstration, le sprint planning etc.

\section{Lexique de VSC Technologies}

\paragraph{Note de livraison: }
\label{lexi:delivery_note}
Une note de livraison est un fichier qui fait le lien entre la vue technique et la vue applicative.

\paragraph{Trigramme: }
\label{lexi:trigramme}
Un trigramme est une application qui a une abréviation de trois lettres.

\chapter{Contexte}
\label{chap:premierchapitre}

\section{LE GROUPE VSC \& Rail Europe} VSC Technologie / Voyages-sncf(oui.sncf) / SNCF
\paragraph{} Le Groupe VSC et Rail Europe, filiale du Groupe SNCF, dirigée par Franck Gervais, est un acteur majeur du tourisme, expert de la distribution du train mais aussi de la vente des billets d'avions, de séjours, location de voitures et chambres d'hôtel, en France et en Europe. En 2015, son volume d’affaires atteint 4,1 milliards d’euros, en recul de 1,4\% par rapport à 2015 en vendant 86 millions de voyages, en croissance de 4,4\%. L’innovation demeure un axe central et exprime la capacité de Voyages-sncf.com à répondre aux nouveaux usages de ses clients. Aujourd’hui, en France, Voyages-sncf.com est le premier site d’e-commerce et la première agence de voyages en ligne ; le groupe rassemble 1200 collaborateurs dans le monde dont 40\% à l'international (130 en Europe et 350 hors Europe).
\paragraph{} En 2000, le site internet Voyages-sncf.com est lancé sur le périmètre France. L’entreprise est à l’époque le distributeur unique des billets de train SNCF, et a pour objectif de transformer son site en portail de voyages offrant des produits et services complémentaires au train. L’année suivante, Voyages-sncf.com forme une joint-venture avec l’américain Expedia et devient une agence de voyage globale. L’entreprise poursuit alors son développement en France, tout en nourrissant une ambition internationale.
Pour répondre aux enjeux de la distribution du voyage et aux nouveaux comportements d’achats, le groupe VSC offre à ses clients mondiaux un réseau puissant, souple et adapté à leurs besoins. Il couvre plus présent dans 11 pays européens et 45 dans le reste du monde via un total de 67 sites internet et mobiles, 4 boutiques et un service de call-center. Afin de répondre aux enjeux spécifiques du marché B2B, le site Voyages-sncf.eu a été lancé en Europe en 12 langues (hors France).
Le site recense plusieurs transporteurs tels que SNCF, TER, Eurostar, Thalys, TGV Lyria ; 3 compagnies de bus, 400 compagnies aériennes ; 280 000 hôtels référencés ; plus de 25 000 offres de séjours ; 30 loueurs de voitures, etc.
\clearpage

\section{Usine Logicielle \& Katana} (à quoi ça sert NDL)

\href{https://wiki.vsct.fr/display/KTN/KATANA+Accueil}{Katana}

\href{https://wiki.vsct.fr/pages/viewpage.action?spaceKey=KTN\&title=Usine+Logicielle+VSCT}{Katana2}

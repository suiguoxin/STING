\chapter{Conclusion}
\label{chap:Conclusion}

\section{Bilan organisation}
J'ai eu la chance de travailler en autonomie et d'être soutenu par une équipe qui a su ma laisser la liberté d'organiser mon travail selon mon rythme,
sans fortes contraintes de temps ni de cahier des charges ou de contrat. J'ai donc pu bénéficier de l'organisation transverse de l'équipe et
ainsi pratiquer les méthodes de développement agile.

\section{Bilan technique}
Grâce à ce projet, j'ai pu avoir ma première expérience en tant que développeur full-stack.

J'ai pu utiliser les différents frameworks du côté serveur et du côté client (Grails 3, Bootstrap 4, jQuery, jQueryUI).
Cela m'a permis de mettre en application mes compétences sur des langages tels que HTML, CSS, JavaScript ou encore Java, mais aussi de découvrir de nouvelles technologies
telles que Bash, SCSS et Groovy.

Au niveau de la gestion du code, j'ai su monter en compétences sur git au travers de méthodes d'organisation sur les branches ou encore les "merge requests".

Egalement, j'ai pu découvrir les différents processus utilisés en entreprises. Cela m'a permis de participer au déploiement manuel de l'application
puis à la mise en place de son intégration continue.

Enfin, j'ai su gagner en confiance en apprennant à répondre à des problématiques au travers de recherches dans les documentations officielles des langages, des frameworks, des plugins
ou encore des forums.

\section{Bilan personnel}
Ce stage a été ma première expérience de travail dans une grande entreprise informatique. Ce fût très enrichissant car il m’a permis de découvrir les aspects
de cette entreprise et d'expérimenter les processus de développement et de déploiement. J'ai ainsi pu mettre en pratique mes connaissances acquises au cours de
ma formation mais surtout j'ai pu les enrichir.

Grâce à cette expérience, je saurai mieux appréhender les projets informatiques et mettre en application tout ce que j'ai pu y découvrir.
Cela me servira dans le cadre de la poursuite de mes études. 

\chapter{Conclusion}
\label{chap:Conclusion}

\section{Bilan personnel}
C'était ma première expérience de travailler dans une grande entreprise informatique.
Ce stage a été très enrichissant pour moi,
car il m’a permis de découvrir les aspects d'une entreprise informatique et expérimenter le processus de développement et de déploiement.

\section{Bilan technique}
Grâce à ce projet, j'ai pu avoir ma première expérience en tant que développeur full-stack.

J'ai pu appliquer les différents frameworks de côté serveur et de côté client (Grails 3, Bootstrap 4, jQuery, jQueryUI).
J'ai pu pratiquer les langages que je connaisais (HTML, CSS, JavaScript, Java)
et j'ai eu la chance d'utiliser les langages que je ne connais pas (Bash, SCSS, Groovy) et les normes récentes (ECMAScript 6).
J'ai pu expérimenter l'environnement open-source.

Au niveau de gestion des codes, j'ai pris une basique connaissance de git et les méthodes d'organisation des branches et des merge requestes.

J'ai pu participer au déploiement manuel de l'application et ensuite à la réalisation de l'intégration continue.
J'ai pu observer la façon de travailler des spécialistes et ensuite créé un petit bout de scripts en Bash comme ce qu'ils ont fait.

Ce qui est le plus important,
je me forme progressivement l'habitude de chercher des informations dans les documentations officielles des frameworks, des plugins et des langages,
je commence à comprendre comment chercher mes questions précises dans les moteurs de recherche comme "Google" et les communautés comme "Stack Overflow",
je me sens plus capable qu'avant de trouver les problèmes et les résoudre.

\section{Bilan organisation et académique}
J'ai eu la chance de travailler en autonomie et d'être soutenu par une équipe.
Le contexte était contraignant au niveau de temps, cahier des charges et contrat,
j'ai eu la liberté d'organiser mon travail à mon rythme.
J'ai bénéficié l'organisation transverse de l'équipe et j'ai pu pratiquer la méthode agile du développement.
C'état une bonne expérience pour moi.

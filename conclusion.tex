\chapter{Conclusion}
\label{chap:Conclusion}

\section{Bilan personnel}
C'était ma permière expérience de travailler dans une grande entreprise informatique.
Ce stage a été très enrichissant pour moi car
il m’a permis de découvrir les aspects d'une entreprise informatique et experimenter le processus de développement.

\section{Bilan technique}
Grâce à ce project, j'ai pu avoir ma permière expérience en taut que développeur full-stack.

J'ai pu appliqué les differents frameworks de coté serveur et de coté client(Grails 3, Bootstrap 4, jQuery, jQueryUI).
J'ai pu pratiquer les languages que je connaisais(HTML, CSS, Javascript, Java)
et j'ai eu la chance d'utiliser les languages que je ne connais pas (Bash, SCSS, Groovy) et les normes récents(ECMAScript 6).
J'ai pu experimenté l'environment open-source.

Au niveau de gestion des codes, j'ai pris une basique connaissance de git et les méthodes d'organisation des branches et des merge requestes.

J'ai pu participé au déploiement manuel de l'application et ensuite au réalisation de l'intégration continue.
J'ai pu observé la façon de travailler des spécialites et ensuite créé un petit bout de scripts comme ce qu'ils ont fait.

Ce qui est le plus important,
je me forme progressivement l'habitude de chercher des informations dans la documentation officielles des frameworks, des plugins et des languages,
je commence à comprendre comment chercher mes questions précises dans les moteurs de recherche comme "Google" et les communauté comme "Stack Overflow",
je me sens plus capable qu'avant de trouver les problèmes et les résoudre.

\section{Bilan organisation et académique}
J'ai eu la chance de travailler en autonaumie et d'être supporté par une équipe.
Le contexte était contraignant au niveau de temps, cahier des charge et contrat,
j'ai eu la liberté d'organiser mon travail.
J'ai bénéficié l'organisation transverse de l'équipe et j'ai pu pratiqué la méthode agile du développement.
